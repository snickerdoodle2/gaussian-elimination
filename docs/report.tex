\documentclass[11pt]{article}
\usepackage[a4paper]{geometry}
\usepackage[parfill, skip=10pt]{parskip}

\usepackage[polish]{babel}
\usepackage[T1]{fontenc}

%\usepackage{amsmath}

%\usepackage{graphicx}
%\graphicspath{ {./assets/} }

\begin{document}
\textbf{Dominik Pilipczuk} \hfill 24.12.2023

\textit{Teoria Wspołbieżności}
\section{Opis problemu}
Zadanie domowe polegało na zaimplementowaniu algorytmu eliminacji Gaussa,
wraz z uwspółbieżnieniem części, która doprowadza do postaci macierzy trójkątnej
górnej.
\section{Rozwiązanie}
Rozwiązanie zostało wykonane z wykorzystaniem języka \texttt{C}
wraz z biblioteką \texttt{OpenMP}. Opis rozwiązania jest w postaci komentarzy
w kodzie.
\subsection{Uruchamianie programu}
By uruchomić program, należy skompilować kod poleceniem:
\begin{verbatim}
   gcc -fopenmp -o main main.c matrix.c 
\end{verbatim}
Następnie uruchomić plik \texttt{main} i wkleić macierz do przetworzenia
(format taki sam jak w treści zadania).
\section{Komentarz}
Program działa poprawnie, jednak dla małych macierzy nie widać dobrze efektów
uwspółbieżnienia. By zobaczyć który wątek wykonuje daną operację, należy zmienić
wartość definicji \texttt{DEBUG} na \texttt{false} w pliku \texttt{main.c}.
\end{document}
