\documentclass[11pt]{article}
\usepackage[a4paper]{geometry}
\usepackage[parfill, skip=10pt]{parskip}

\usepackage[polish]{babel}
\usepackage[T1]{fontenc}

%\usepackage{amsmath}

\usepackage{graphicx}
\graphicspath{ {./assets/} }

\begin{document}
\textbf{Dominik Pilipczuk} \hfill 24.12.2023

\textit{Teoria Wspołbieżności}
\section{Opis problemu}
Zadanie domowe polegało na zaimplementowaniu algorytmu eliminacji Gaussa,
wraz z uwspółbieżnieniem części, która doprowadza do postaci macierzy trójkątnej
górnej.
\section{Rozwiązanie}
Rozwiązanie zostało wykonane z wykorzystaniem języka \texttt{C}
wraz z biblioteką \texttt{OpenMP}. Opis rozwiązania jest w postaci komentarzy
w kodzie.
\subsection{Uruchamianie programu}
By uruchomić program, należy skompilować kod poleceniem:
\begin{verbatim}
   gcc -fopenmp -o main main.c matrix.c 
\end{verbatim}
Następnie uruchomić plik \texttt{main} i wkleić macierz do przetworzenia
(format taki sam jak w treści zadania).

\newpage
\section{Dowód Rozwiązania}
Algorytm eliminacji Gaussa polega na usuwaniu elementów pod przekątną wiersz
po wierszu, przez co można go łatwo zrównoleglić. Wystarczy, że dla każdego
wiersza wykonamy następujące kroki:
\begin{enumerate}
    \item Wybór punktu zerowego (elementu na przekątnej)
    \item Obliczenie współczynnika $\frac{a_{(j+x),j} }{a_{j,j}}$, gdzie j to numer wiersza
    \item Pomnożenie wiersza przez współczynnik z kroku 2
    \item Odjęcie wiersza od pozostałych wierszy, tak by w kolumnie punktu
        zerowego były same zera
\end{enumerate}

Kroki 3 i 4 można wykonać równolegle dla każdej kolumny, przez co graf zależności
wygląda tak:

\begin{center}
    \includegraphics[width=.7\textwidth]{graph.png}

    \textbf{Rysunek 1.} Graf zależności dla macierzy $3$x$3$.
\end{center}

Gdzie wierzchołek $A$ to krok 1 oraz 2, wierzchołek $B$ to krok 3,
a wierzchołek $C$ to krok 4. Dla macierzy o większych rozmiarach,
graf zależności będzie wyglądał podobnie, jednak będzie bardziej rozbudowany.
\section{Komentarz}
Program działa poprawnie, jednak dla małych macierzy nie widać dobrze efektów
uwspółbieżnienia. By zobaczyć który wątek wykonuje daną operację, należy zmienić
wartość definicji \texttt{DEBUG} na \texttt{false} w pliku \texttt{main.c}.
\end{document}
